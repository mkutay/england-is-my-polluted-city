\documentclass[10pt, a4paper]{scrartcl}
\renewcommand{\baselinestretch}{1.15}

\usepackage{changepage}
\usepackage{multicol}
\usepackage{paralist}
\usepackage{listings}
\usepackage{amsmath}
\usepackage{hyperref}
\usepackage[lmargin=0.8in, rmargin=0.8in, tmargin=1in, bmargin=1.3in]{geometry}

\setkomafont{disposition}{\normalfont\bfseries}
\parskip=8pt

\begin{document}

    \pagenumbering{gobble}

    \begin{titlepage}
        \begin{center}
            \LARGE
            \textbf{KING'S COLLEGE LONDON}

            \vspace{2cm}

            \begin{adjustwidth}{-1cm}{-1cm}
                \centering
                \Large
                \textbf{4CCS1PPA PROGRAMMING PRACTICE AND APPLICATIONS}
            \end{adjustwidth}

            \vspace{0.5cm}

            \Large
            \textbf{Fourth "Air Pollution" Coursework (Mar 2025)}

            \vspace{2cm}

            \Large
            Project Name: Englang is My Polluted City

            \vspace{1cm}

            \Large
            \begin{tabular}{l l}
                Student Name: & Mehmet Kutay Bozkurt \\
                Student ID: & 23162628 \\
                \vspace{0.5cm} & \\
                Student Name: & Anas Ahmed \\
                Student ID: & 23171444 \\
                \vspace{0.5cm} & \\
                Student Name: & Matthias Loong \\
                Student ID: & 23078800 \\
                \vspace{0.5cm} & \\
                Student Name: & Chelsea Feliciano \\
                Student ID: & 22042916
            \end{tabular}
        \end{center}
        % \tableofcontents
    \end{titlepage}

    \begin{multicols}{2}

        \pagenumbering{arabic}

        \section{Introduction}
        

        \section{Directions for Use}

        \section{Tasks Lists and Implementation Details}

        \subsection{Base Tasks}

        \noindent \textbf{Welcome Panel:}

        \noindent \textbf{Data Visualisation Panel (Map View):} We have implemented a colour scheme for the entire map, where the colour
        of a grid area is determined by the pollution
        level of that area. The colour scheme is implemented as an interface \verb|ColourScheme| with a single method \verb|getColour()|.
        This allows us to have different colour schemes for the map, and a way to switch between them quite easily — improving 
        maintainability and responsibility-driven design.

        \noindent \textbf{Pollution Statistics Panel:}

        \noindent \textbf{Detailed Grid Data:}

        \subsection{Unit Testing}

        \noindent Unit testing was implemented for various classes and various methods within those classes. The unit tests are
        implemented with JUnit. The tests were implemented to test the functionality of the methods in the classes, and to ensure
        that the methods are working as expected. These tests were vital in adding functionality to the codebase, as they allowed
        us to ensure that the new functionality did not break any existing functionality.

        \subsection{Challenge Tasks}

        \noindent \textbf{More Extensive Graph Based Trends:}

        \noindent \textbf{Interactive Map:}

        \noindent \textbf{Adding the Entire UK in the Map:}

        \noindent \textbf{Various Optimisations:} Various optimisations were made to the codebase to vastly improve performance.
        LOD (Level of Detail) was implemented to decrease the number of grid areas rendered on the map, and the number of data points
        rendered in the graph as the user zooms out. Additionally, culling was implemented to only render grid areas that are visible
        on the screen.
       
        \section{Code Quality Considerations}

        \subsection{Coupling and Responsibility-Driven Design}
      
        \subsection{Cohesion}
        
        \subsection{Maintainability}
        
        \section{Final Remarks}
        
    \end{multicols}

\end{document}