\documentclass[10pt, a4paper]{scrartcl}
\renewcommand{\baselinestretch}{1.15}

\usepackage{changepage}
\usepackage{multicol}
\usepackage{paralist}
\usepackage{listings}
\usepackage{amsmath}
\usepackage{hyperref}
\usepackage{xcolor}
\usepackage[lmargin=0.8in, rmargin=0.8in, tmargin=1in, bmargin=1.3in]{geometry}

\setkomafont{disposition}{\normalfont\bfseries}
\parskip=8pt

% \RedeclareSectionCommand[
%     beforeskip=-0.75\baselineskip,
%     afterskip=0.75\baselineskip]{section}

% \RedeclareSectionCommand[
%     beforeskip=-0.75\baselineskip,
%     afterskip=0.75\baselineskip]{subsection}

\begin{document}
    
\pagenumbering{gobble}

\begin{titlepage}
    \begin{center}
        \Huge
        \textbf{KING'S COLLEGE LONDON}

        \vspace{2cm}

        \LARGE
        \textbf{4CCS1PPA PROGRAMMING PRACTICE AND APPLICATIONS}

        \vspace{1cm}

        \Large
        \textbf{Fourth “Air Pollution” Coursework (Mar 2025)}

        \vspace{1cm}

        \Large
        Project Name: England is My Polluted City

        \vspace{1cm}

        \Large
        \begin{tabular}{l l}
            Student Name: & Mehmet Kutay Bozkurt \\
            Student ID: & 23162628 \\
            \vspace{0.5cm} & \\
            Student Name: & Anas Ahmed \\
            Student ID: & 23171444 \\
            \vspace{0.5cm} & \\
            Student Name: & Matthias Loong \\
            Student ID: & 23078800 \\
            \vspace{0.5cm} & \\
            Student Name: & Chelsea Feliciano \\
            Student ID: & 22042916
        \end{tabular}
    \end{center}
    % \tableofcontents
\end{titlepage}

\begin{multicols}{2}

\pagenumbering{arabic}

\section{Introduction}

\noindent This project implements a fully functional JavaFX Application to display historical Air Pollution Data of the whole
United Kingdom on an interactive map. Users are able to explore the historic pollution levels between 2018 and 2023
on a colour coded map, as well as view detailed location and pollution data at specific data points. There are many
quality of life features implemented, such as a robust GUI to filter through different years and pollutants, an extensive 
statistics framework, as well as a “colourblind mode” to make the pollution viewer more accessible. 

\noindent This project fulfils all of the base requirements and extends functionality greatly by implementing features such as
Live Air Quality API integration and a highly detailed statistical analysis tool. The challenge extensions are
listed in greater detail in the report (see section \ref{challenge-tasks}).

\noindent Additionally, this project takes great care and consideration for high cohesion and low coupling in the codebase,
making all classes modular and following the Model View Controller design pattern to fully separate UI
functionalities from the logic of the program.

\section{Directions for Use}

\noindent The main application class \verb|App| can be found inside the app folder, and running it as a JavaFX application
starts the programme. Additional libraries are used in the programme to increase the usability of the application.
These libraries are \href{https://github.com/gluonhq/maps}{Gluon Maps} (and its dependencies),
\href{https://github.com/dstl/osgb}{OSGB}, \href{https://github.com/geographiclib/geographiclib-java}{GeographicLib},
and \href{https://github.com/google/gson}{Gson}.
All of these libraries are already inside the given BlueJ project.

\section{Roles and Responsibilities Breakdown}

\noindent Overall, the codebase was equally worked upon by the entire team. Contributions are demonstrated in the
commit history best, but the following outlines the main tasks each member worked on.

\subsection{Mehmet Kutay Bozkurt}

\vspace{-0.2em}

\begin{itemize}
    \itemsep-0.3em 
    \item Created the statistics framework, created and added all of the statistical calculators and UI panels. Created the
    statistics controller to go through different panels.
    \item Created pop-up on the map to view detailed information about location.
    \item Created the API that accesses the \href{https://postcodes.io/}{Postcodes.io} end-point to get detailed information
    about the location, such as region, borough, and constituency.
    \item Created the singleton class \verb|DataManager| to be the only entry point to the pollution data.
    \item Started the colouring framework for the map pollution values.
\end{itemize}

\vspace{-0.2em}

\subsection{Anas Ahmed}

\vspace{-0.2em}

\begin{itemize}
    \itemsep-0.3em 
    \item Pollution layer: Rendering of pollution polygons onto the map, handling scaling, sizing, mapping squares onto spherical
    world-map and optimisation with LODs (Level Of Details).
    \item Pollution threshold system.
    \item Improvements and small refactors in colour system and legend.
    \item Major refactors throughout the integration of UI and backend elements (model-view system).
\end{itemize}

\vspace{-0.2em}

\subsection{Matthias Loong}

\vspace{-0.2em}

\begin{itemize}
    \itemsep-0.3em 
    \item Implemented the \verb|DataPicker| class to dynamically select a data set according to the pollutant and year with the
    GUI side panel and \verb|ComboBox| selectors.
    \item Implemented Real-Time Air Quality Index API.
    \item Updated Information Pop-Up to display more detailed information at a given data point.
    \item Created About and Welcome Page views.
\end{itemize}

\vspace{-0.2em}

\subsection{Chelsea Feliciano}

\vspace{-0.2em}

\begin{itemize}
    \itemsep-0.3em 
    \item Created collapsible legend pane.
    \item Added colourblind colour scheme.
    \item Created zoom Control Buttons — Zoom In/Out and Fullscreen.
    \item Created the cat button to toggle side panel visibility.
    \item Created and modified the design and layout of UI components (apart from statistics related stuff).
\end{itemize}

\vspace{-0.2em}

\section{Base Tasks}

\subsection{Welcome Panel}

\noindent A welcome panel was implemented as an additional window that greets users when launching the app. Users can
scroll through screenshots and visual instructions on how to use the app effectively. If needed, the welcome screen is
easily accessible again by clicking on “Help” and then “Tutorial” on the navigation bar.
    
\noindent Error handling was implemented for image file paths in the case they get deleted or are entered incorrectly
in the codebase, when the screen is being updated.
    
\noindent Additionally, an “About” panel was implemented that gives detailed acknowledgements to the authors of the
app along with credits for the external libraries and APIs used.
    
\subsection{Data Visualisation Panel (Map View)}

\noindent The map was implemented using Gluon Maps, an external maps library created by the maintainers of JavaFX.
The map displays all pollution data provided from the CSV files and colour codes them on the map. While the map starts
with a view of London, the user is able to drag around the map to take a closer look at specific areas of London by
zooming in, or view the entirety of the UK by zooming out. This can be done with the scroll wheel or with the zoom
buttons on the upper-right hand corner of the screen.
    
\noindent A new \verb|PollutionLayer| class was created to process colours and visualise the pollution data on the map accurately.
For smoother performance, Level of Detail optimisation was implemented. More information about the optimisations are
detailed in the “Optimisations” section of this report.
    
\noindent Users are able to view a specified pollutant and year, in addition to being able to change the
colours of the map to be colorblind friendly.
    
\noindent Useful buttons on the right side of the map allow users to toggle a legend to view the values of pollution
in relation to the colours layered on the map, as well as allowing users to switch between full-screen and windowed mode.
    
\noindent A colour scheme for the map was implemented where the colour of a grid area is determined by its pollution
level. The colour scheme is implemented as an interface \verb|ColourScheme| with a single method \verb|getColour|.
This allows the application to have different colour schemes for the map, and a way to switch between them quite easily,
improving maintainability and responsibility-driven design.

\subsection{Pollution Statistics Panel}

\noindent Various statistical information can be accessed by clicking the “View Pollutant Statistics” on the side panel.
Each type of information is divided into different panels, which can be viewed through using the “Next” and “Previous”
buttons on the bottom. Different panels include

\vspace{-0.2em}

\begin{itemize}
    \itemsep-0.3em
    \item “Pollution Hotspots Trends” for recording the highest, median, and the lowest pollution levels with their locations.
    This panel also integrates with the Postcode API to showcase where the area is located.
    
    \item“Average Pollution Trends” for showing how the pollution values changed over time with the mean, median, and
    standard deviation being shown for each year. Additional information is also displayed for the overall averages and
    percent changes.
    
    \item “All Pollutions” for seeing how much the average value for each pollutant has changed over the years.
\end{itemize}

\vspace{-0.2em}
    
\noindent The statistics module follows a three-tier pattern: Calculators (backend logic for statistical computations),
result objects (intermediate representation for just the data), and panels (visualisation and user interface). Each one
of these follows extensibility in mind. Specifically, an abstract class is used for the general \verb|StatisticsPanel|
class, which each different type of statistics views \verb|extend|. Each result object and calculator also implement a
similar interface called \verb|StatisticsResult| and \verb|StatisticsCalculator|, respectively. With the
\verb|StatisticsPanelFactory| and \verb|StatisticsManager| classes, adding new statistics panels is easy and requires no
changes to other parts of the code, showing the high level of cohesion and low coupling.
    
\noindent The actual graph charts were also divided into components for any panel to use. \verb|LineChartPanel| expects a
variable argumant for the data, allowing additional lines with different colours to be added easily. These components were
created for high modularity.
    
\noindent Finally, the UI panels and the backend calculators are not coupled at all. The main view is managed by the
\verb|StatisticsController| which calls the \verb|StatisticsManager| to generate the results using the calculators; then,
the controller supplies the results to the \verb|StatisticsPanelFactory|, which in return creates the actual panel views.

\subsection{Detailed Grid Data}

\noindent When right-clicking anywhere on the map, users are able to view a pop-up with all relevant information for a data
point, such as: Latitude and Longitude Coordinates, Unique Grid Code, Pollution Level for the selected Pollutant and Year,
Borough or County, and nearest UK Postcode. 
    
\noindent The grid data also implements an API (explained in-depth in section \ref{api-integration}) to retrieve real-time
air quality updates and displays it accordingly.
    
\noindent Users can click anywhere on the map to close the pop-up. Right clicking another location on the map will
close the current pop-up and open a new one with the correct information for the selected data point.

\section{Unit Testing}

\noindent Unit testing was implemented with JUnit for various classes and public methods in those classes to test the
\textit{interface} of those classes. The tests ensure that the methods are working as expected and use different forms
of input ranging from valid to invalid. For example, for querying location data, both valid coordinates within London
and invalid coordinates were tested to ensure successful querying and correct error handling. Tests were also created
on all of the statistics calculators to ensure that they give out correct information. This was done by calculating statistical
values differently to how they are in the calculators. Overall, these tests were vital in adding functionality to the codebase,
as they ensured that new functionality did not break any existing functionality.

\section{Challenge Tasks} \label{challenge-tasks}

\subsection{Interactive Map}

\noindent The GluonMaps external library was used to implement an interactive map. It features tile-based maps, allowing for easy
management of zooming in and out and dragging of the map within the view. Furthermore, the library features robust
information, allowing us to convert locations on the screen to coordinates for easy mapping of the data provided.
    
\subsection{Adding the Entire UK in the Map}

\noindent Given that the project was already utilising an external mapping library and the data provided was for the whole
United Kingdom, we decided to expand the coverage of the project to account for all data points.
    
\subsection{API Integration} \label{api-integration}

\noindent The project makes use of two external APIs to display information related to the data provided:

\begin{itemize}
    \itemsep-0.3em
    \item The Postcodes API allows the usage of provided coordinates to return details about the selected data points:
    UK Postal Code, County/Borough, and Country. 
    
    \item The AQICN API was used to get real-time air-quality updates from the World Air Quality Index and display the values
    to the user along with the time the Air Quality Index was last updated.
\end{itemize}
    
\subsection{More Extensive Graph Based Trends}

\noindent Two additional panels were added for more statistical information on the pollution data. These are:

\vspace{-0.1em}

\begin{itemize}
    \itemsep-0.3em
    \item “Histogram” for seeing the distribution of pollution value ranges—displayed on a bar chart with a logarithmic scale.
    Additionally, specific values can be seen by hovering the bars to open a tooltip.
    
    \item “Distribution Analysis” to showcase the percentiles over time and some more statistical values of the data
    (such as skewness and kurtosis). Hovering over the table shows what these specific values mean.
\end{itemize}

\vspace{-0.1em}
    
\noindent These panels were added using the framework mentioned above, where two additional components were added (\verb|DataTablePanel|
and \verb|HistogramChartPanel|). Tooltips for each cell in the table from \verb|DataTablePanel| was added to explain the meaning
of the values. For \verb|HistogramChartPanel|, an additional \verb|LogarithmicAxis| class was added to show a logarithmic scale
for better viewing.

\noindent Overall, these modular components enhanced how the statistics information is shown.

\subsection{Pollution Thresholds}

\noindent Users are able to use a slider that will display areas with a certain pollution percentage and other areas with a gray colour.

\section{Code Optimisations}

Various optimisations were made to the codebase to vastly improve performance. These include:

\begin{itemize}
    \itemsep-0.3em
    \item LOD (Level of Detail) was implemented to decrease the number of grid areas rendered on the map, and the number of
    data points rendered in the graph as the user zooms out. 
    
    \item Polygon view-culling was implemented to only render grid areas that are visible on the screen.
    
    \item Multithreadding was added to fetch the LOD values, data from the CSV files, and generate the statistics for
    multiple years. This allowed all of the values to be loaded and generated in parallel, increasing the efficiency.
\end{itemize}

\section{Bugs and Issues}

Most issues were identified and fixed thanks to rigorous testing, as mentioned above, but some issues were identified
and could not be fixed due to time constraints:

\begin{itemize}
    \itemsep-0.3em
    \item Loading Pollution Statistics for the first time before caching will take a few seconds, causing some lag. All
    subsequent loading of the page will be instant.
    
    \item When zooming in and out of the map view quickly, due to the loading of the LODs, the map stops for a couple
    moments. This is due to the loading of the LODs and the rendering of the polygons, which was deemed unavoidable.

    \item Opening the About or Welcome page when in fullscreen might result in unexpected behaviour, which seems to be 
    a JavaFX bug only happening on MacOS.
        
    \item Most issues were identified and fixed thanks to rigorous testing, as mentioned above.
\end{itemize}

\section{Libraries used}

\noindent Various APIs and external libraries were used in this project for extended functionality:

\begin{itemize}
    \itemsep-0.3em
    \item \href{https://github.com/gluonhq/maps}{Gluon Maps}: Maps library that implements OpenStreetMaps.

    \item \href{https://github.com/dstl/osgb}{OSGB by DST}: Library to convert Easting and Northing to
    Latitude and Longitude.

    \item \href{https://github.com/geographiclib/}{GeographicLib}: Used for geodesic distance calculation.

    \item \href{https://github.com/google/gson}{Gson}: A library by Google to serialise JSON to
    Java objects.

    \item \href{https://postcodes.io/}{Postcodes.io}: API used to get location and address data.

    \item \href{https://aqicn.org/api/}{World Air Quality Index API}: API used to get real time air quality index updates.
\end{itemize}
    
\end{multicols}
\end{document}